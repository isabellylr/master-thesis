% !TEX root = ../thesis.tex

\chapter{Conclusion}
\label{sec:conclusion}

Psychological experiments, as Byrne's suppression task and Wason's selection task, have shown that human deviate from classical logic when performing reasoning tasks. considering this, we have focused on a non-monotonic approach based on the weak completion of logic programs to formalise episodes of human reasoning. More precisely, we have worked with a connectionist network which encodes such an approach and generates the sceptical consequences of an abductive problem.

This network consists of three main components which are responsible for generating the candidate explanations, computing the consequences of each explanation, and, finally, deriving the sceptical consequences for the given observation. Deciding whether a formula follows sceptically from an abductive framework is DP-complete, a complexity that is outside of NP. Because the bottleneck of this problem is the exponential number of candidate explanations which has to be considered, we have focused on the component of the network responsible for their generation and proposed some ways of optimising it.

In Chapter~\ref{sec:cn} we have presented the current state of the network which generates all the non-complementary candidate explanations. Later in this chapter we have shown one possible way to reduce the number on candidate explanations which is introducing the minimality constraint. In this chapter, we have considered a static sequence of candidate explanations. However, the large amount of possible sequences in which these candidate explanations can be considered is a strong indicator that humans do not always consider a static predefined sequence.

Based on this, in Chapter~\ref{sec:nn} we have proposed to substitute the current approach by a recurrent network, such as Jordan or Elman, which has shown to perform perfectly in the task of learning arbitrary sequences of candidate explanation. These networks have presented an error rate of zero in all the cases tested and we have shown how general this approach is, allowing us to easily adapt to generate different sequences of non-complementary, minimal and even bounded candidate explanations.

The complexity of this task gives us the strong impression that humans do not consider all the candidate explanations but rather a subset of it. One possible way of reducing this number is by applying the minimality constraint which we have previously discussed. However, this minimality implies a cardinality order which we have not verified. Besides this, even if the minimality constraint reduces the number of candidates in practice, the complexity of the problem remains the same in the worst case. Therefore, we strongly believe that there are other parameters which are taken into account when generating the candidate explanation and we define a bound based on them.

Considering this, we have proposed the setup of some psychological experiments in Chapter~\ref{sec:exp} which verify the assumptions we have made and also try to identify some other properties in the ordering and bounds of the candidate explanations. As mentioned before, the new approach proposed here is designed in such a way that it could be aligned with these experiments independent on their outcome.

As future work, we propose the realisation of the psychological experiments discussed earlier. If the results of these experiments do not confirm the assumptions we have made, then we also propose to apply the necessary modifications to our approach such that it is aligned with the cognitive process regarding the task of solving an abductive problem. Moreover, the experimental results might also allow us to add more constraints to the process of generating candidate explanations, concerning for example the bound or ordering of the candidates.

Summing up, we have come up with an approach for the generation of candidate explanations which can be easily adapted to arbitrary sequences. As discussed before, from the psychological point of view, there are still many open questions concerning sceptical abduction. Therefore, having such a versatile approach is the ideal solution considering the current state of the problem.